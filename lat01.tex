\documentclass[10pt]{article}

%\usepackage{epsf}
%\usepackage{epsfig}
\usepackage{hyperref}
%\usepackage[usenames]{color}

%\newcommand{\eat}[1]{}
%\newcommand{\heading}[1]{
%\noindent
%\vspace{-2pt}
%\hspace{-10pt}
%{%\color{Gray}
%\noindent \rule{\linewidth}{0.3mm}}
%\noindent {\scshape #1}
%\vspace{-7pt}
%\\
%{\color{Gray}
%\rule{\linewidth}{0.05mm}}
%}
%
%\newcommand{\subheading}[3]{
%%\hspace{-.14in}
%\begin{tabular*}{6.64in}{l@{\extracolsep{\fill}}r}
%\textbf{#1}. {\em #2}. & #3 %\\
%%\textit{#2} &  \\
%\end{tabular*}}

\usepackage{fullpage}
\addtolength{\textheight}{5cm}
\addtolength{\textwidth}{1cm}
\addtolength{\topmargin}{-2cm}
\pagestyle{empty}


\begin{document}
{\Huge {\sc Niranjan Viladkar}}
\hfill
\parbox{180pt}{\raggedleft
	{\bf Sr. Data Scientist} \\
	\href{http://www.fractalanalytics.com/} {Fractal Analytics}, \\
	Bangalore,{\sc India} \\
	\vspace{5pt} \noindent \\
	Phone: +91-9663339655 \\
	Email: \href{mailto:viladkar.niranjan@gmail.com}{viladkar.niranjan@gmail.com} \\
	Web: \href{https://sites.google.com/site/niranjanviladkar/} {Personal}, \href{https://www.linkedin.com/in/niranjan-viladkar-47aa3139/}{LinkedIn} \\
	Last updated on : \today
}
%employment
\vspace{15pt}
\hrule
\vspace{5pt}
{\Large \bfseries \slshape \sc Employment}
\vspace{5pt}
\hrule
\vspace{10pt}

\parbox{90pt}{\raggedright
	{\bf Sr. Data Scientist} \\
	Fractal Analytics, \\
	Bangalore,\\
	India. \\
	August'16 onwards  \\
	(1 Year 8 months)
}
\parbox{390pt}{
	\begin{itemize}
	\item Working (as a vendor) with marketing analytics team of a leading online advertisements MNC.
              I am involved in projects that handle several hundred thousands to millions of unique avertisers across globe to solve problems 
              such as `Who are prospects for up-sell / cross-sell?', `Revenue impact estimation of campaigns, product features, etc.' 
        \item Worked (in earlier projects) on problems of type anomaly detection, employee behaviour segmentation and predictive modelling.
	\item Worked in domains of CPG, Security, Online Advertisements.
	\item Initiated capability as part of AI and ML team on `Emotion Recognition from Audio'
        \item Core technologies - bash shell scripting, SQL, Python and R. Also completed several internal assessments related to these.
	\end{itemize}
}
\vspace{10pt}


\parbox{90pt}{\raggedright
	{\bf Research Engineer}\\
	Xerox Research Center India\\
	Bangalore,\\ 
	India.\\
	July'14 - August'16\\
	(2 Years)
}
\parbox{390pt}{
	\begin{itemize}
	\item Optimized Keyword Spotting System - Used shared memory multi threading SIMD instructions, Intel IPP vector operation optimizations to make the system sub-real time. (C/C++)
	\item Built a cloud based real time conversational (Apple Siri like) agent - Involved acoustic and language model generation using of Kaldi speech recognition tool-kit, secure TCP/IP websocket communication between speech recogniser server, text-to-speech server (Python, shell scripting, core Java).
	\item Built a pronunciation predicting decision tree based dictionary - Given a spelling, returns a pronunciation in the form of phonemes. Involved writing python scripts to run various stages of input processing and decision tree learning.(Python, shell scripting)
	\item (Briefly) worked with Theano deep learning tool-kit to experiment hashing techniques in building a phoneme recogniser - Involved handling a very deep \& large network, high dimensional \& large data-set, highly sparse space with minimal memory storage requirement O(32GB). (Python, Git versioning tool)
	\end{itemize}
}
\vspace{10pt}

\parbox{90pt}{\raggedright
	{\bf Teaching Assistant} \\
	Deptt. of Computer Science \\
	IIT Delhi, India. \\
	July'12 - June'14 \\
	(Academic 2 Years)
}
%\hfill
\parbox{390pt}{
	\begin{itemize}
	\item Conducted labs, prepared \& evaluated assignments and exam question papers for 4 semesters at IIT Delhi.
	\item Won `Outstanding Teaching Assistant Award'. \href{http://www.cse.iitd.ac.in/index.php/2011-12-29-23-16-01/teaching-assistant-awards}{Link}
	\item Comment from the Course coordinator, Prof. Saroj Kaushik: Niranjan was very enthusiastic, concerned about improvement of the students' knowledge, and helped with designing assignments and MOSS.
	\end{itemize}		
	
}
\vspace{10pt}

\parbox{90pt}{\raggedright
	{\bf Software Developer} \\
	Oracle, Bangalore,\\
	India.\\
	June'11 - June'12\\
	(1 Year)
}
%\hfill
\parbox{390pt}{
	\begin{itemize}
	\item Worked as Member Technical Staff on the product - Oracle Social Network.
	\item Involved in server side API and performance testing.
	\item Worked in Java and JUnit for test automation.
	\end{itemize}		
	
}

\pagebreak 
\_
\vspace{25pt}

\hrule
\vspace{3pt}
{\Large \bfseries \slshape \sc Publications}
\vspace{3pt}
\hrule
\begin{itemize}
\item Co-inventor on Xerox patent (submitted in Aug'16) - Discriminative DNN Hashing Technique for High Dimensional and Massive Scale Machine Learning 
\item Niranjan Viladkar, Vivek Tyagi, Arunasish Sen, Sriranjani R, Pragathi Praveena, ``Xerox Conversational AI Agent (XCAI) for Enterprise Knowledge-base Q\&A", Show \& Tell Industrial Research Track, IEEE ICASSP 2016 \url{http://www.icassp2016.org/ST-3.asp}. Demo video - \url{https://goo.gl/aEuqEU}
\item Niranjan Viladkar, Vivek Tyagi, ``Real time Large Vocabulary Speech Recognition and Keyword Spotting", Xerox Innovation Group Research Conference at Webster, Rochester, NY, September, 2015. \href{https://drive.google.com/open?id=0ByWGijDf3JffS2lrZmw0Mkdwc0U}{Link to poster} 
\end{itemize}

%education
\hrule \vspace{3pt}
{\Large \bfseries \slshape \sc Education}
\vspace{3pt}\hrule\vspace{5pt}


\parbox{130pt}{
	{\bf Master of Technology} \\
	Deptt. of Computer Science \\
	IIT Delhi, India. \\
	July'12 - June'14 \\
	GPA: 9.03 out of 10
}
\parbox{85pt}{
	{\bf Udacity Nano degree} \\
	Machine Learning (Basic) \\
	March'18
}
\parbox{130pt}{
	{\bf Bachelor of Technology} \\
	Deptt. of Computer Science \\
	NIT Nagpur, India. \\
	July'07 - May'11 \\
	GPA: 7.88 out of 10
}
\parbox{120pt}{
	{\bf Higher Secondary} (Grade 12) \\
	Maharashtra State Board \\
	R.Y.K College of Science \\
	Nashik, India. \\
	July'04 - March'06 \\
	Percentage : 78\%
}
\vspace{10pt}

\hrule \vspace{3pt}
{\Large \bfseries \slshape \sc Projects}
\vspace{3pt} \hrule \vspace{7pt}

\parbox{0.97\textwidth}{
{\bf Masters' Thesis} - May 2014 \\
{\bf Title} - Incorporating Object and People Information to Improve Video Activity Recognition \\
{\bf Guide} - Dr. Parag Singla, IIT Delhi.\\
{\bf Details} - The project combines the first order logic and probability to improve the accuracy of prediction of human activity recognition classically done using machine learning and/or object recognition techniques. Achieved absolute 5\% improvement over baseline. (Java, Matlab, Shell scripting)
}
\vspace{7pt}

\parbox{0.97\textwidth}{
{\bf Bachelors' Thesis} - May 2011 \\
{\bf Title} - Heap Reference Analysis and Its Implementation in GCC \\
{\bf Guide} - Dr. Uday Khedker, IIT Bombay and Dr. C. S. Moghe, VNIT Nagpur.\\
{\bf Details} - Worked in a team of 3 on heap reference analysis in GCC back-end towards improving conventional garbage collection. Wrote a GCC Inter-Procedural Analysis (IPA) pass, which works on liveness of memory location rather than just the reachability for garbage collection decision. (C, GNU Make)
}

\vspace{7pt}

\parbox{0.97\textwidth}{
{\bf Course Project} - May 2013 \\
{\bf Title} - Parallelised SVD Computation over CPU and GPU \\
{\bf Guide} - Dr. Subodh Kumar, IIT Delhi.\\
{\bf Details} - Worked in a team of 2 to implement a parallel version of singular value decomposition of a matrix using CUDA over GPU and C/C++ over CPU. Achieved speed ups proportional to the size of input matrix. Parallel version runs 4.4x faster over serial version for input matrix of the size 4000x4000.
}

\vspace{10pt}

%\hrule
%\vspace{3pt}
%{\Large \bfseries \slshape \sc Technical Skills and Tool[-kit]s}
%\vspace{3pt}
%\hrule
%\begin{itemize}
%\item Bash shell scripting, Python, C/C++, Core Java, HTML, JavaScript, CUDA, can learn new technology quickly.
%\item Git/svn versioning, Eclipse IDE, Linux, GIMP image editor, Kaldi, Theano
%\end{itemize}

\vspace{6pt}
\hrule
\vspace{3pt}
{\Large \bfseries \slshape \sc Awards}
\vspace{3pt}
\hrule
\begin{itemize}
\item Astronomy Olympiad, 2006 -  Selected in National Astronomy Olympiad organised by HBCSE, TIFR. Was
amongst top 45 students all over India.
\item OPJEMS Scholar, 2007 - Awarded O P Jindal Engineering and Management Scholarship. Was amongst top 20
students all over India. Link : \url{http://www.opjems.com/opjems_scholars.html} \textgreater 2007 \textgreater NIT Nagpur. 
\end{itemize}

%\vspace{6pt}
\hrule
\vspace{3pt}
{\Large \bfseries \slshape \sc Passion}
\vspace{3pt}
\hrule
\begin{itemize}
\item I love working with systems and coding, be it linux, GCC compiler source code or project codebase. I enjoyed my stints from B.Tech till Xerox where my work required heavy and (almost) daily coding/building up systems. (Said this, I also have lot to pick up to be a professional)
\item To my system skills, I am able to add analytics and machine learning skills via projects at Fractal.
\item I strongly believe these two skills go hand in hand for building a system with high impact.
\end{itemize}

\end{document}
